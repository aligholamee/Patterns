\documentclass[12pt]{article}
\usepackage{latexsym,amssymb,amsmath} % for \Box, \mathbb, split, etc.
% \usepackage[]{showkeys} % shows label names
\usepackage{cite} % sorts citation numbers appropriately
\usepackage{path}
\usepackage{url}
\usepackage{verbatim}
\usepackage[pdftex]{graphicx}

% horizontal margins: 1.0 + 6.5 + 1.0 = 8.5
\setlength{\oddsidemargin}{0.0in}
\setlength{\textwidth}{6.5in}
% vertical margins: 1.0 + 9.0 + 1.0 = 11.0
\setlength{\topmargin}{0.0in}
\setlength{\headheight}{12pt}
\setlength{\headsep}{13pt}
\setlength{\textheight}{625pt}
\setlength{\footskip}{24pt}

\renewcommand{\textfraction}{0.10}
\renewcommand{\topfraction}{0.85}
\renewcommand{\bottomfraction}{0.85}
\renewcommand{\floatpagefraction}{0.90}

\makeatletter
\setlength{\arraycolsep}{2\p@} % make spaces around "=" in eqnarray smaller
\makeatother

% change equation, table, figure numbers to be counted inside a section:
\numberwithin{equation}{section}
\numberwithin{table}{section}
\numberwithin{figure}{section}

% begin of personal macros
\newcommand{\half}{{\textstyle \frac{1}{2}}}
\newcommand{\eps}{\varepsilon}
\newcommand{\myth}{\vartheta}
\newcommand{\myphi}{\varphi}

\newcommand{\IN}{\mathbb{N}}
\newcommand{\IZ}{\mathbb{Z}}
\newcommand{\IQ}{\mathbb{Q}}
\newcommand{\IR}{\mathbb{R}}
\newcommand{\IC}{\mathbb{C}}
\newcommand{\Real}[1]{\mathrm{Re}\left({#1}\right)}
\newcommand{\Imag}[1]{\mathrm{Im}\left({#1}\right)}

\newcommand{\norm}[2]{\|{#1}\|_{{}_{#2}}}
\newcommand{\abs}[1]{\left|{#1}\right|}
\newcommand{\ip}[2]{\left\langle {#1}, {#2} \right\rangle}
\newcommand{\der}[2]{\frac{\partial {#1}}{\partial {#2}}}
\newcommand{\dder}[2]{\frac{\partial^2 {#1}}{\partial {#2}^2}}
\usepackage{enumitem}
\newcommand{\nn}{\mathbf{n}}
\newcommand{\xx}{\mathbf{x}}
\newcommand{\uu}{\mathbf{u}}
\usepackage{tikz}
\usetikzlibrary{arrows}
\usetikzlibrary{positioning}
\usepackage{titlesec}
\newcommand{\junk}[1]{{}}
\usepackage{xcolor}
\definecolor{darkblue}{rgb}{0,0,0.4}
\usepackage[colorlinks = true,
linkcolor = darkblue,
urlcolor  = darkblue,
citecolor = darkblue,
anchorcolor = darkblue]{hyperref}
% set two lengths for the includegraphics commands used to import the plots:
\newlength{\fwtwo} \setlength{\fwtwo}{0.45\textwidth}
% end of personal macros

\begin{document}
\DeclareGraphicsExtensions{.jpg}

\begin{center}
\textsc{\Large Statistical Pattern Recognition} \\[2pt]
	\textsc{\large Assignment 1}\\
	\vspace{0.5cm}
  Ali Gholami \\[6pt]
  Department of Computer Engineering \& Information Technology\\
  Amirkabir University of Technology  \\[6pt]
  \def\UrlFont{\em}
  \url{http://ceit.aut.ac.ir/~aligholamee}\\
    \href{mailto:aligholamee@aut.ac.ir}{\textit{aligholamee@aut.ac.ir}}
\end{center}

\begin{abstract}
This is an introductory assignment to the world of \textit{Statistics} and \textit{Probability} in the world of \textit{Pattern Recognition}. We'll introduce some key concepts like \textit{Probability Distribution Function, Cumulative Distribution Function, Probability Density Function, Probability Mass Function, Joint Probability Density Function, Joint Cumulative Density Function, Marginal Density} \& more details as the probabilistic point of view. Furthermore, we'll review the concepts of \textit{Expected Value, Variance, Standard Deviation, Covariance \& Correlation of Random Variables(e.g. Random Vectors), Univariate \& Multivariate Gaussian Distribution, Total Probability \& Bayes Theorem, Geometric \& Mahalanobis Distances, Central Limit Theorem, Independence \& Correlation} as the statistics point of view. Also, a principal concept called \textit{Linear Transformation} is discussed. The relationship between these fields is far more important than each separately.
\end{abstract}

\subparagraph{Key Words.} \textit{PDF, PMF, JPDF, JPMF, CDF, JCDF, Covariance Matrix, Correlation Coefficient, Correlation, Variance, Expected Vector, Gaussian Distribution, Marginal Probability, Linear Tranformation, Eigenvector, Eigenvalue, Rank.}

\section{Practical \& Theoretical Problems}
\subsection{1. Expectation \& Variance}

A random variable \textit{$X$} has \textit{$E(X) = -4$} and \textit{$E(X^2) = 30$.} Let \textit{$Y = -3X + 7$.} Compute the following.

\begin{enumerate}[label=(\alph*)]
	\item \textit{$V(X)$}
	
	\item \textit{$V(Y)$}
	
	\item \textit{$E((X+5)^2)$}
	
	\item \textit{$E(Y^2)$}
	
\end{enumerate}

\titleformat{\section}
{\normalfont\Large\bfseries}   % The style of the section title
{}                             % a prefix
{0pt}                          % How much space exists between the prefix and the title
{Section \thesection:\quad}    % How the section is represented

\subsection{Solution}

The main equation to calculate the \textit{Variance} of a random variable \textit{X} is given in \ref{eq:1.1}.
\begin{equation}\label{eq:1.1}
	V(X) = E[(X - E[X])^2]
\end{equation}
Expanding the equation \ref{eq:1.1}, we'll have the equation \ref{eq:1.2} using simple calculus.

$$
	V(X) = E[X^2 + E[X]^2 -2XE[X]]
$$
$$
	V(X) = E[X^2] + E[X]^2 - 2E[X]^2
$$
\begin{equation}\label{eq:1.2}
V(X) = E[X^2] - E[X]^2
\end{equation}
\noindent\rule{\textwidth}{.5pt}
(a) The equation \ref{eq:1.2} can be conducted directly to compute the \textit{Variance}. Replacing the values from the problem description we get the following as result.
$$
	V(X) = E[X^2] - E[X]^2 
$$
$$
	V(X) = 30 - 16 = 14
$$
\noindent\rule{\textwidth}{.5pt}
(b) It is important to mention the \textit{Linearity} of \textit{Expectation} operator as formally described in \ref{eq:1.3}.
\begin{equation}\label{eq:1.3}
	E[aX + b] = aE[X] + b
\end{equation}
Using property  \ref{eq:1.3}, we can write the \textit{V(Y)} as

$$
	V(Y) = V(-3X + 7) = E[(-3X + 7)^2] - E[-3X + 7]^2
$$
$$
	V(Y) = E[9X^2 + 49 -42X] - E[-3X + 7]^2
$$
$$
	V(Y) = 9E[X^2] + E[49] - 42E[X] - 9E[X]^2 - E[49]	
$$
$$
	V(Y) = 9*30 + 49 - 42*(-4) - 9*30 - 49
$$
$$
	V(Y) = 168
$$
\end{document}