\documentclass[12pt]{article}
\usepackage{latexsym,amssymb,amsmath} % for \Box, \mathbb, split, etc.
% \usepackage[]{showkeys} % shows label names
\usepackage{cite} % sorts citation numbers appropriately
\usepackage{path}
\usepackage{url}
\usepackage{verbatim}
\usepackage[pdftex]{graphicx}

% horizontal margins: 1.0 + 6.5 + 1.0 = 8.5
\setlength{\oddsidemargin}{0.0in}
\setlength{\textwidth}{6.5in}
% vertical margins: 1.0 + 9.0 + 1.0 = 11.0
\setlength{\topmargin}{0.0in}
\setlength{\headheight}{12pt}
\setlength{\headsep}{13pt}
\setlength{\textheight}{625pt}
\setlength{\footskip}{24pt}

\renewcommand{\textfraction}{0.10}
\renewcommand{\topfraction}{0.85}
\renewcommand{\bottomfraction}{0.85}
\renewcommand{\floatpagefraction}{0.90}

\makeatletter
\setlength{\arraycolsep}{2\p@} % make spaces around "=" in eqnarray smaller
\makeatother

% change equation, table, figure numbers to be counted inside a section:
\numberwithin{equation}{section}
\numberwithin{table}{section}
\numberwithin{figure}{section}

% begin of personal macros
\newcommand{\half}{{\textstyle \frac{1}{2}}}
\newcommand{\eps}{\varepsilon}
\newcommand{\myth}{\vartheta}
\newcommand{\myphi}{\varphi}

\newcommand{\IN}{\mathbb{N}}
\newcommand{\IZ}{\mathbb{Z}}
\newcommand{\IQ}{\mathbb{Q}}
\newcommand{\IR}{\mathbb{R}}
\newcommand{\IC}{\mathbb{C}}
\newcommand{\Real}[1]{\mathrm{Re}\left({#1}\right)}
\newcommand{\Imag}[1]{\mathrm{Im}\left({#1}\right)}

\newcommand{\norm}[2]{\|{#1}\|_{{}_{#2}}}
\newcommand{\abs}[1]{\left|{#1}\right|}
\newcommand{\ip}[2]{\left\langle {#1}, {#2} \right\rangle}
\newcommand{\der}[2]{\frac{\partial {#1}}{\partial {#2}}}
\newcommand{\dder}[2]{\frac{\partial^2 {#1}}{\partial {#2}^2}}
\usepackage{enumitem}
\newcommand{\nn}{\mathbf{n}}
\newcommand{\xx}{\mathbf{x}}
\newcommand{\uu}{\mathbf{u}}
\usepackage{tikz}
\usetikzlibrary{arrows}
\usetikzlibrary{positioning}
\usepackage{titlesec}
\newcommand{\junk}[1]{{}}
\usepackage{sectsty}
\usepackage{xcolor}

\makeatletter
\renewcommand*\env@matrix[1][\arraystretch]{%
	\edef\arraystretch{#1}%
	\hskip -\arraycolsep
	\let\@ifnextchar\new@ifnextchar
	\array{*\c@MaxMatrixCols c}}
\makeatother

\makeatletter
\renewcommand*\env@matrix[1][*\c@MaxMatrixCols c]{%
	\hskip -\arraycolsep
	\let\@ifnextchar\new@ifnextchar
	\array{#1}}
\makeatother

\definecolor{darkblue}{rgb}{0,0,0.4}
\usepackage[colorlinks = true,
linkcolor = darkblue,
urlcolor  = darkblue,
citecolor = darkblue,
anchorcolor = darkblue]{hyperref}
% set two lengths for the includegraphics commands used to import the plots:
\newlength{\fwtwo} \setlength{\fwtwo}{0.45\textwidth}
% end of personal macros

\begin{document}
\DeclareGraphicsExtensions{.jpg}

\begin{center}
\textsc{\Large Statistical Pattern Recognition} \\[2pt]
	\textsc{\large Assignment 1}\\
	\vspace{0.5cm}
  Ali Gholami \\[6pt]
  Department of Computer Engineering \& Information Technology\\
  Amirkabir University of Technology  \\[6pt]
  \def\UrlFont{\em}
  \url{http://ceit.aut.ac.ir/~aligholamee}\\
    \href{mailto:aligholamee@aut.ac.ir}{\textit{aligholamee@aut.ac.ir}}
\end{center}

\begin{abstract}
This is an introductory assignment to the world of \textit{Statistics} and \textit{Probability} in the context of \textit{Pattern Recognition}. We'll introduce some key concepts like \textit{Probability Distribution Function, Cumulative Distribution Function, Probability Density Function, Probability Mass Function, Joint Probability Density Function, Joint Cumulative Density Function, Marginal Density} \& more details as the probabilistic point of view. Furthermore, we'll review the concepts of \textit{Expected Value, Variance, Standard Deviation, Covariance \& Correlation of Random Variables(e.g. Random Vectors), Univariate \& Multivariate Gaussian Distribution, Total Probability \& Bayes Theorem, Geometric \& Mahalanobis Distances, Central Limit Theorem, Independence \& Correlation} as the statistics point of view. Also, a principal concept called \textit{Linear Transformation} is discussed. The relationship between these fields is far more important than each separately.
\end{abstract}

\subparagraph{Keywords.} \textit{PDF, PMF, JPDF, JPMF, CDF, JCDF, Covariance Matrix, Correlation Coefficient, Correlation, Variance, Expected Vector, Gaussian Distribution, Marginal Probability, Linear Tranformation, Eigenvector, Eigenvalue, Rank.}

\section{Expectation \& Variance}

A random variable \textit{$X$} has \textit{$E(X) = -4$} and \textit{$E(X^2) = 30$.} Let \textit{$Y = -3X + 7$.} Compute the following.

\begin{enumerate}[label=(\alph*)]
	\item \textit{$V(X)$}
	
	\item \textit{$V(Y)$}
	
	\item \textit{$E((X+5)^2)$}
	
	\item \textit{$E(Y^2)$}
	
\end{enumerate}

\titleformat{\section}
{\normalfont\Large\bfseries}   % The style of the section title
{}                             % a prefix
{0pt}                          % How much space exists between the prefix and the title
{Section \thesection:\quad}    % How the section is represented

\subsection*{Solution}

The main equation to calculate the \textit{Variance} of a random variable \textit{X} is given in \ref{eq:1.1}.
\begin{equation}\label{eq:1.1}
	V(X) = E[(X - E[X])^2]
\end{equation}
Expanding the equation \ref{eq:1.1}, we'll have the equation \ref{eq:1.2} using simple calculus.

$$
	V(X) = E[X^2 + E[X]^2 -2XE[X]]
$$
$$
	V(X) = E[X^2] + E[X]^2 - 2E[X]^2
$$
\begin{equation}\label{eq:1.2}
V(X) = E[X^2] - E[X]^2
\end{equation}
\noindent\rule{\textwidth}{.5pt}
(a) The equation \ref{eq:1.2} can be conducted directly to compute the \textit{Variance}. Replacing the values from the problem description we get the following as result.
$$
	V(X) = E[X^2] - E[X]^2 
$$
$$
	V(X) = 30 - 16 = 14
$$
\noindent\rule{\textwidth}{.5pt}
(b) It is important to mention the \textit{Linearity} of \textit{Expectation} operator as formally described in \ref{eq:1.3}.
\begin{equation}\label{eq:1.3}
	E[aX + b] = aE[X] + b
\end{equation}
Using property  \ref{eq:1.3}, we can write the \textit{V(Y)} as

$$
	V(Y) = V(-3X + 7) = E[(-3X + 7)^2] - E[-3X + 7]^2
$$
$$
	V(Y) = E[9X^2 + 49 -42X] - E[-3X + 7]^2
$$
$$
	V(Y) = 9E[X^2] + E[49] - 42E[X] - 9E[X]^2 - E[49]	
$$
$$
	V(Y) = 9*30 + 49 - 42*(-4) - 9*30 - 49
$$
$$
	V(Y) = 168
$$
\noindent\rule{\textwidth}{.5pt}
(c) Expanding the internals of the expectation, we'll get the following.
$$
	E[(X+5)^2] = E[X^2 + 10X + 25]
$$
$$
	E[(X+5)^2] = E[X^2] + 10E[X] + E[25]
$$
$$
	E[(X+5)^2] = 30 + 10*(-4) + 25
$$
$$
	E[(X+5)^2] = 15
$$
\noindent\rule{\textwidth}{.5pt}
(c) Same as above, we'll use 1.2 to get the following.\\
$$
	E[Y^2] = E[(3X + 7)^2]
$$
$$
	E[Y^2] = E[9X^2 + 49 - 42X]
$$
$$
	E[Y^2] = 487
$$

\section{Eigenvector \& Eigenvalue}
(a) Compute eigenvalues and eigenvectors of
$
A = 
	\renewcommand\arraystretch{1}
	\setlength\arraycolsep{6pt}
\begin{bmatrix}
	 4 & 0 & 0 \\
	 0 & 2 & 2 \\
	 0 & 9 & -5 \\
\end{bmatrix}
$
and compare your results with Matlab outputs.
\vspace{.5cm}
\\
(b) A $2*2$ matrix \textit{A} has $\lambda_1 = 2$ and $\lambda_2 = 5$, with corresponding eigenvectors 
$
V_1 = 
	\renewcommand\arraystretch{1}
	\setlength\arraycolsep{6pt}
\begin{bmatrix}
1 & 0
\end{bmatrix}^\mathsf{\textit{T}}
$
and
$
V_2 = 
	\renewcommand\arraystretch{1}
	\setlength\arraycolsep{6pt}
\begin{bmatrix}
1 & 1
\end{bmatrix}^\mathsf{\textit{T}}
$. Find \textit{A}.

\titleformat{\section}
{\normalfont\Large\bfseries}   % The style of the section title
{}                             % a prefix
{0pt}                          % How much space exists between the prefix and the title
{Section \thesection:\quad}    % How the section is represented

\subsection*{Solution}
(a) First of all, we'll find the \textit{eigenvalues} using the \textit{characteristic equation} given in \ref{eq:2.1}.

\begin{equation}\label{eq:2.1}
| A - \lambda I |= 0
\end{equation}
Using this equation, all of the diagonal components of matrix \textit{A} will be decremented by a $\lambda$ term. The determinant of the resulting matrix will be as following.
$$
	A - \lambda I = 
		\renewcommand\arraystretch{1}
		\setlength\arraycolsep{6pt}
	\begin{bmatrix}
	4-\lambda & 0 & 0 \\
	0 & 2-\lambda & 2 \\
	0 & 9 & -5-\lambda \\
	\end{bmatrix}	
$$
$$
| A - \lambda I | = (4-\lambda)(\lambda^2 + 3\lambda - 28)
$$
This will result in the following two roots for the $\lambda$ term.
$$
	\boxed{\lambda_1 = 4} \ \ \ \boxed{\lambda_2 = -7}
$$
These are the \textit{eigenvalues} of the given matrix \textit{A}. We'll continue using the \textit{Gaussian Elimination} technique to compute the \textit{eigenvectors} of matrix \textit{A}. According to the equation \ref{eq:2.2} we are looking for all possible vectors that can be substitute with vector X.
\begin{equation}\label{eq:2.2}
| A - \lambda I |X= 0
\end{equation}
Thus, for each \textit{eigenvalue} determined in the previous computations, we'll find the proper \textit{eigenvector} by converting the equation \ref{eq:2.2} to a \textit{Row Echelon Form} and solving the resulting linear system by \textit{Back Substitution}. Using the computed \textit{eigenvalues}, we'll have \ref{eq:2.3} and \ref{eq:2.4}
\begin{equation}\label{eq:2.3}
	A - 4I = 
		\renewcommand\arraystretch{1}
		\setlength\arraycolsep{6pt}
		\begin{bmatrix}
		0 & 0 & 0 \\
		0 & -2 & 2 \\
		0 & 9 & -9 \\
		\end{bmatrix}
\end{equation}

\begin{equation}\label{eq:2.4}	
	A + 7I = 
		\renewcommand\arraystretch{1}
		\setlength\arraycolsep{6pt}
		\begin{bmatrix}
		11 & 0 & 0 \\
		0 & 9 & 2 \\
		0 & 9 & 2 \\
		\end{bmatrix}
\end{equation}
Now we can find the proper \textit{X} for each \textit{eigenvalue}, using \textit{augmented} version of \ref{eq:2.3} and \ref{eq:2.4}. We'll have \ref{eq:2.5} and \ref{eq:2.6} as result.
\begin{equation}\label{eq:2.5}
	E_1 = (A - 4I \ | \ 0)= 
	\renewcommand\arraystretch{1}
	\setlength\arraycolsep{6pt}
		\begin{bmatrix}[ccc|c]
		0 & 0 & 0 & 0\\	
		0 & -2 & 2 & 0\\
		0 & 9 & -9 & 0\\
		\end{bmatrix}
\end{equation}

\begin{equation}\label{eq:2.6}	
	E_2 = (A + 7I \ | \ 0) = 
	\renewcommand\arraystretch{1}
	\setlength\arraycolsep{6pt}
		\begin{bmatrix}[ccc|c]
		11 & 0 & 0 & 0\\	
		0 & 9 & 2 & 0\\
		0 & 9 & 2 & 0\\
		\end{bmatrix}
\end{equation}
The above matrices will be converted to the \textit{Row Echelon Form} below using \textit{Row Operations.}
$$
	E_1 = 
		\renewcommand\arraystretch{1}
		\setlength\arraycolsep{6pt}
		\begin{bmatrix}[ccc|c]
		0 & 9 & -9 & 0\\	
		0 & -2 & 2 & 0\\
		0 & 9 & -9 & 0\\
		\end{bmatrix}
$$
$$
	E_1 = 
		\renewcommand\arraystretch{1}
		\setlength\arraycolsep{6pt}
		\begin{bmatrix}[ccc|c]
		0 & 9 & -9 & 0\\	
		0 & -2 & 2 & 0\\
		0 & 0 & 0 & 0\\
		\end{bmatrix}
$$
$$
	E_1 = 
	\renewcommand\arraystretch{1}
	\setlength\arraycolsep{6pt}
	\begin{bmatrix}[ccc|c]
	0 & 1 & -1 & 0\\	
	0 & -2 & 2 & 0\\
	0 & 0 & 0 & 0\\
	\end{bmatrix}
$$
$$
E_1 = 
\renewcommand\arraystretch{1}
\setlength\arraycolsep{6pt}
\begin{bmatrix}[ccc|c]
0 & 1 & -1 & 0\\	
0 & 0 & 0 & 0\\
0 & 0 & 0 & 0\\
\end{bmatrix}
$$
The resulting equation for $E_1$ will be as following.
$$
	(0)(X_1) + (1)(X_2) - (1)(X_3) = 0
$$
Thus, we'll have the following \textit{eigenvector} and \textit{eigenvalue}.
$$
	X = 
		\renewcommand\arraystretch{1}
		\setlength\arraycolsep{6pt}
		\begin{bmatrix}[ccc|c]
		X_1\\	
		X_2\\
		X_3\\
		\end{bmatrix} = \renewcommand\arraystretch{1}
		\setlength\arraycolsep{6pt}
		\begin{bmatrix}[ccc|c]
		0\\	
		X_2\\
		X_2\\
		\end{bmatrix}
$$
$$
	X =  X_2 \renewcommand\arraystretch{1}
	\setlength\arraycolsep{6pt}
	\begin{bmatrix}[ccc|c]
	0\\	
	1\\
	1\\
	\end{bmatrix} , \ \ \lambda = 4
$$
Calculating the \textit{eigenvector} using the same procedure, we'll get the following results.
\begin{equation}\label{eq:2.7}	
E_2 = (A + 7I \ | \ 0) = 
\renewcommand\arraystretch{1}
\setlength\arraycolsep{6pt}
\begin{bmatrix}[ccc|c]
11 & 0 & 0 & 0\\	
0 & 9 & 2 & 0\\
0 & 9 & 2 & 0\\
\end{bmatrix}
\end{equation}
Using \textit{Row Operations} we'll have:
$$
E_2 = 
\renewcommand\arraystretch{1}
\setlength\arraycolsep{6pt}
\begin{bmatrix}[ccc|c]
11 & 0 & 0 & 0\\	
0 & 9 & 2 & 0\\
0 & 0 & 0 & 0\\
\end{bmatrix}
$$
Since $E_2$ is in the \textit{Row Echelon Form}, we can infer the equation for $E_2$ as:
$$
(11)(X_1) + (1)(X_2) = 0
$$
For the \textit{eigenvector} we'll have:
$$
	X =  X_2 \renewcommand\arraystretch{2}
	\setlength\arraycolsep{10pt}
	\begin{bmatrix}[ccc|c]
	\frac{-1}{11}\\	
	1\\
	0\\
	\end{bmatrix} , \ \ \lambda = -7
$$
\noindent\rule{\textwidth}{.5pt}
(b) Since \textit{A} is a $2*2$ matrix, we have:
$$
	A = 
	\renewcommand\arraystretch{1}
	\setlength\arraycolsep{6pt}
	\begin{bmatrix}
	a & b\\	
	c & d\\
	\end{bmatrix}
$$
Given \textit{eigenvectors} and \textit{eigenvalues} in the problem description, we can conduct the \textit{Characteristic Equation} presented in \ref{eq:2.2} and get the following results.
$$
(A - 2 I)X = 
\renewcommand\arraystretch{1}
\setlength\arraycolsep{6pt}
\begin{bmatrix}
a-2 & b\\	
c & d-2\\
\end{bmatrix}
\begin{bmatrix}
1\\
0 \\
\end{bmatrix} = \begin{bmatrix}
0\\
0 \\
\end{bmatrix}
$$
$$
(A - 5 I)X = 
\renewcommand\arraystretch{1}
\setlength\arraycolsep{6pt}
\begin{bmatrix}
a-5 & b\\	
c & d-5\\
\end{bmatrix}
\begin{bmatrix}
1\\
1 \\
\end{bmatrix} = \begin{bmatrix}
0\\
0 \\
\end{bmatrix}
$$
Simply solving these equations will give us the following results.
$$
	\boxed{a = 2}
	\ \ \
	\boxed{b = 3}
	\ \ \
	\boxed{c = 0}
	\ \ \
	\boxed{d = 5}
$$
Finally, we'll rewrite the matrix \textit{A} replacing the new parameters in it.
$$
A = 
\renewcommand\arraystretch{1}
\setlength\arraycolsep{6pt}
\begin{bmatrix}
2 & 3\\	
0 & 5\\
\end{bmatrix}
$$
\newpage
\section{Probability Density Function}
Let
\renewcommand\arraystretch{1.2}
\setlength\arraycolsep{10pt}
\[f(x, y) = \left\{
\begin{array}{lr}
c(x + y^2) &  if \ \ 0 < x < 1\ ;\ 0 < y < 1\\
0 &  elsewhere.
\end{array}
\right.
\]
Find each of the following.

\begin{enumerate}[label=(\alph*)]
	\item \textit{$c$}
	
	\item \textit{$f_X (x)$}
	
	\item \textit{$f_{X|Y} (y)$}
	
	\item Are \textit{X} and \textit{Y} independent?
	
	\item What is the probability $Pr(X<\frac{1}{2}  | Y = \frac{1}{2})
	$?
	 
\end{enumerate}

\end{document}